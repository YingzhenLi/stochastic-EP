

\section{Expectation Propagation and Assumed Density Filtering}
%short review on EP
We begin by briefly reviewing the EP and ADF algorithms upon which our new method is based. Consider for simplicity observing a dataset comprising $N$ i.i.d.~samples $\mathcal{D} = \{\bm{x}_n \}_{n=1}^N$ from a probabilistic model $p(\bm{x}|\bm{\theta})$ parametrised by an unknown $d$-dimensional vector $\bm{\theta}$. Exact Bayesian inference involves computing the posterior distribution of the parameters given the data, 
\begin{equation}
p(\bm{\theta} | \mathcal{D}) \propto p_0(\bm{\theta}) \prod_{n=1}^{N} p(\bm{x}_n | \bm{\theta}).
\end{equation}
%
Typically, the posterior is intractable and requires approximation. A standard application of EP constructs such an approximation using a simpler distribution $q(\bm{\theta}) \in \mathcal{Q}$ comprising a product of simple factors,
%
% and from a distribution family $\mathcal{P}$, is introduced to estimate the underlying data distribution. Bayesian methods require posterior computation after observing dataset $D = \{\bm{x}_i \}_{i=1}^N$ using Bayes Rule: 
%\begin{equation}
%p(\bm{\theta} | D) \propto p_0(\bm{\theta}) \prod_{i=1}^{N} p(\bm{x}_i | \bm{\theta}),
%\end{equation}
%however this posterior is often in some intractable family $\tilde{\mathcal{P}}$ for many powerful probabilistic models. Expectation propagation approximates the true posterior with a suitable distribution $q(\bm{\theta}) \in \mathcal{Q}$, which factorises over likelihood terms
%
\begin{equation}
q(\bm{\theta}) \propto p_0(\bm{\theta}) \prod_{n=1}^{N} f_n(\bm{\theta}).
\end{equation}
%
The objective is to refine the approximate factors so that they capture the contribution of each of the likelihood terms to the posterior i.e.~$f_n(\bm{\theta}) \approx p(\bm{x}_n | \bm{\theta})$. In this spirit, one approach would be to iteratively update each approximating factor $f_n(\bm{\theta})$ by minimising $\mathrm{KL}[p(\bm{\theta}|\mathcal{D}) || p(\bm{\theta}|\mathcal{D}) f_n(\bm{\theta})/ p(\bm{x}_n | \bm{\theta})]$, but this is intractable as it involves computing the full posterior. Instead, EP approximates this procedure using a four simple steps. First, the factor selected for update is removed from the approximation to produce the so-called cavity distribution, $q_{-n}(\bm{\theta}) =q(\bm{\theta})/f_n(\bm{\theta})$. Second, the corresponding likelihood is included to produce the tilted distribution $\tilde{p}_n(\bm{\theta}) = q(\bm{\theta})/f_n(\bm{\theta}) p(\bm{x}_n | \bm{\theta})$. Third EP updates the approximating factor by minimising $\mathrm{KL}[\tilde{p}_n(\bm{\theta}) || q_{-n}(\bm{\theta})  f_n(\bm{\theta})]$. The hope is that the contribution the true-likelihood makes to the posterior is similar to the effect the same likelihood has on the tilted distribution. If the approximating distribution is in the exponential family, as is often the case, then the KL minimisation is equivalent to a moment-matching step \cite{amari:ig} that we denote $f_i(\bm{\theta}) \leftarrow \mathtt{proj}[\tilde{p}_i(\bm{\theta})] / q_{-i}(\bm{\theta}) $. Finally, having updated the factor, it is included into the approximating distribution.\todo[fancyline]{mention damping here?}
%
%Next EP updates the  $\mathrm{KL}(p(\bm{\theta})||q(\bm{\theta}))$
%
% is removed from the approximating distribution
%
%
%by matching the moments of the corresponded single datapoint posterior. To be precise, an EP iteration begins with removing the selected factor $f_i(\bm{\theta})$ to form the cavity distribution $q_{-i}(\bm{\theta})$, then uses it as the prior distribution to incorporate the current likelihood $p(\bm{x}_i| \bm{\theta})$. The local posterior $\tilde{p}_i(\bm{\theta}) \propto p(\bm{x}_i|\bm{\theta}) q_{-i}(\bm{\theta})$ is also referred as the tilted distribution, which is in the $\tilde{\mathcal{P}}$ family as well. 
%
%Next EP proposes a moment projection (M-projection) \cite{amari:ig} or moment matching step to approximate the tilted distribution by minimising $KL(p(\bm{\theta})||q(\bm{\theta}))$ wrt.~$q(\bm{\theta})$, and finally recovers the new updates of the current factor $f_i(\bm{\theta}) \buildrel\propto\over \leftarrow q(\bm{\theta}) / q_{-1}(\bm{\theta})$.

We summarise the update procedure for a single factor in Algorithm \ref{alg:ep}. Critically, the approximation step of EP involves local computations since one likelihood term is treated at a time. The assumption is that these local computations, although possibly requiring further approximation, are far simpler to handle compared to the full posterior $p(\bm{\theta}| \mathcal{D})$. 

% ADF
Computation of the cavity distribution requires removal of the current approximating factor and this means that any implementation of EP must store them explicitly necessitating an $\mathcal{O}(N)$ memory footprint. One option is to simply ignore the removal step replacing the cavity distribution with the full approximation. This has the advantage that only the global approximation need be maintained in memory, but as the moment matching step now over-counts the underlying approximating factor, $\mathrm{KL}[q(\theta) p(\bm{x}_n | \bm{\theta}) || q(\bm{\theta})]$, the variance of the approximation shrinks to zero as multiple passes are made through the dataset. Early stopping is therefore required to prevent overfitting.

\section{Stochastic Expectation Propagation}

% SEP
EP has been shown very successful in previous investigations as mentioned, however very little work has been done on large datasets due to its large memory consumption. It requires the program to store every local approximator $f_i(\bm{\theta})$, resulting in space complexity $\mathcal{O}(Nd^2)$ if using Gaussians. To eliminate the linear factor $N$ in the storage requirement, we propose a factor-tying approach by defining a new approximation structure
\begin{equation}
q(\bm{\theta}) \propto f(\bm{\theta})^N p_0(\bm{\theta}),
\end{equation}
and run EP by pretending the $N$ copies as independent factors. We sketch the learning procedure for one update in Algorithm \ref{alg:sep}, and refer it as stochastic expectation propagation (SEP) since it incorporates information from a single sample to all the tied factors. In practice memory allocation for $f(\bm{\theta})$ is unnecessary because $f(\bm{\theta}) \propto (q(\bm{\theta}) / p_0(\bm{\theta}))^{\frac{1}{N}}$ and $q_{-1}(\bm{\theta}) \propto q(\bm{\theta})^{1 - \frac{1}{N}} p_0(\bm{\theta})^{\frac{1}{N}}$. Hence our approach with Gaussian factors reduces the storage requirement drastically to $\mathcal{O}(d^2)$, the same as other global approximation algorithms, and enables applications to very large datasets. 

\begin{figure}[!t]
%
\begin{minipage}[t]{0.48\linewidth}
\centering
\begin{algorithm}[H] 
\caption{EP {\color{gray}(and ADF)}} \small
\label{alg:ep} 
\begin{algorithmic}[1] 
	\STATE choose a factor $f_n$ to refine:
	\STATE compute cavity distribution \\$q_{-n}(\bm{\theta}) \propto q(\bm{\theta}) / f_n(\bm{\theta})$ {\color{gray}(ADF: $q_{-n}(\bm{\theta}) = q(\bm{\theta})$)}
	\STATE compute tilted distribution \\$\tilde{p}_i(\bm{\theta}) \propto p(\bm{x}_n|\bm{\theta}) q_{-n}(\bm{\theta})$
	\STATE moment matching: \\$f_n(\bm{\theta}) \leftarrow \mathtt{proj}[\tilde{p}_n(\bm{\theta})] / q_{-n}(\bm{\theta}) $
	\STATE inclusion: $q(\bm{\theta}) \leftarrow q_{-n}(\bm{\theta}) f_n(\bm{\theta})$
\end{algorithmic}
\end{algorithm}
\end{minipage}
\quad \quad
\begin{minipage}[t]{0.45\linewidth}
\centering
\begin{algorithm}[H]
\caption{Stochastic EP} \small
\label{alg:sep} 
\begin{algorithmic}[1] 
%\STATE initialize $\{\tilde{f}_a\}$
	\STATE choose a datapoint $\bm{x}_n\sim D$ to incorporate:
	\STATE compute cavity distribution \\ $q_{-1}(\bm{\theta}) \propto q(\bm{\theta}) / f(\bm{\theta})$
	\STATE compute tilted distribution \\$\tilde{p}_i(\bm{\theta}) \propto p(\bm{x}_n|\bm{\theta}) q_{-1}(\bm{\theta})$
	\STATE moment matching: \\$f_i(\bm{\theta}) \leftarrow \mathtt{proj}[\tilde{p}_i(\bm{\theta})] / q_{-1}(\bm{\theta}) $
	\STATE inclusion: $q(\bm{\theta}) \leftarrow q_{-1}(\bm{\theta}) f_i(\bm{\theta})$
	\STATE \textit{update} $f(\bm{\theta}) \leftarrow f(\bm{\theta})^{1 - \frac{1}{N}} f_i(\bm{\theta})^{\frac{1}{N}}$
\end{algorithmic}
\end{algorithm}
\end{minipage} 
%
\end{figure}

%
We also consider Assumed density filtering (ADF) \cite{maybeck:adf}\cite{minka:ep}, the streaming version of EP, and show its connection to SEP. ADF successively incorporates the incoming datapoints to the approximate posterior by using the posterior computed on previous samples as the current prior. Hence the algorithm only stores the global posterior and is scalable on large datasets. However it always treats the next input as a new observation, making ADF with multiple passes of data flawed in nature. 
%
We re-introduce SEP as to correct ADF by re-scaling the parameters (line 2 in Algorithm \ref{alg:sep}). In this way SEP preserves the advantage of ADF in memory consumption but also retains the correct uncertainty level as full EP. Also multiple passes of datasets helps $q(\bm{\theta})$ ``forget" the bad approximations gradually, making SEP much more robust to observation ordering. 